% !TeX root = 高等微积分.tex
\subsection{反函数求导}
若$f$是连续单射$f\colon D\to \mathbb{R}$, 则$f$有反函数$f^{-1}$, \textbf{问:}$f^{-1}$是否可导? 如何求导?

\begin{proposition}
    若$f$与$f^{-1}$皆可导, 则有
    \begin{equation}
      \left( \mathrm{d} f^{-1} \right) _{f\left( x_0 \right) } = \left( \mathrm{d} f_{x_0} \right) ^{-1}, \quad \left( f^{-1} \right) ' \left( f\left( x_0 \right)  \right) f'\left( x_0 \right) = 1
    \end{equation}
\end{proposition}
\begin{proof}
    由于
    \begin{equation}
      f \circ f^{-1} = \operatorname{id} _{D} ,\quad f^{-1} \circ f = \operatorname{id}_{f(D)}
    \end{equation}
    用链式法则, 有
    \begin{equation}
        \begin{cases}
          \left( \mathrm{d} f^{-1} \right) _{f\left( x_0 \right) } \circ \mathrm{d} f_{x_0} = \operatorname{Id},
          \\
          \mathrm{d} f_{x_0} \circ \left( \mathrm{d} f^{-1} \right) _{f\left( x_0 \right) } = \operatorname{Id}
        \end{cases}
    \end{equation}
    可知
    \begin{equation}
      \left( \mathrm{d} f^{-1} \right) _{f\left( x_0 \right) } = \left( \mathrm{d} f_{x_0} \right) ^{-1}.
    \end{equation}
\end{proof}

\begin{example}
    $f\left( x \right) = x^{3}$, 在$\mathbb{R}$上严格单调, 但$f^{-1}$在$y = 0$处不可导.

    严格单调可导函数的反函数未必可导, 因为若$f^{-1}$在$f\left( x_0 \right) $处可导$\implies f'\left( x_0 \right) \neq 0$.
\end{example}

\begin{theorem}
    设$f\colon D\to \mathbb{R}$是连续单射, 且$D$是区间, 若$f$在$x_0$处可导且$f'\left( x_0 \right) \neq 0$, 则$f^{-1}$在$f\left( x_0 \right) $处可导. 且有
    \begin{equation}
      \left( f^{-1} \right) ' \left( f\left( x_0 \right)  \right)  = \frac{1}{f'\left( x_0 \right) }.
    \end{equation} 
\end{theorem}

\begin{proof}
    由导数的定义计算
    \begin{equation}
      \left( f^{-1} \right) ' \left( f\left( x_0 \right)  \right) = \lim_{y \to f\left( x_0 \right) } \frac{f^{-1}\left( y \right) - f^{-1}\left( f\left( x_0 \right)  \right) }{y - f\left( x_0 \right) }.
    \end{equation}
    用复合极限定理
    ,定义$g\colon D / \{ x_0 \} \to \mathbb{R}, u \mapsto g(u) = \frac{u-x_0}{f\left( u \right) - f\left( x_0 \right) }$, 复合为
    \begin{equation}
      h(y) = g\left( f^{-1}\left( y \right)  \right) = \frac{f^{-1}\left( y \right) - x_0}{y - f\left( x_0 \right) }
    \end{equation}
    注意到
    \begin{equation}
      \lim_{y \to f\left( x_0 \right) } f^{-1}\left( y \right) \xlongequal{\text{已证 $f^{-1}$ 连续}} f^{-1}\left( \lim_{y \to f\left( x_0 \right) } y \right) = x_0.
    \end{equation}
    故有
    \begin{equation}
      \lim_{u \to x_0} g\left( u \right) = \lim_{u \to x_0} \frac{u-x_0}{f\left( u \right) - f\left( x_0 \right) } \xlongequal{\text{四则运算}} \frac{1}{f'\left( x_0 \right) }.
    \end{equation}
    修正方案I自动成立, 由$f$单知$\forall y \neq f\left( x_0 \right) $有$f^{-1}\left( y \right) \neq x_0$.
    这样由复合极限定理得到
    \begin{equation}
      \left( f^{-1} \right) ' \left( f\left( x_0 \right)  \right) = \frac{1}{f' \left( x_0 \right) }.
    \end{equation}
\end{proof}

\noindent
\textbf{推论: }设$f^{-1}$是$f$的反函数, 且$f$是可导函数, 则
\begin{equation}
  \left( f^{-1} \right) ' \left( f\left( x \right)  \right) = \frac{1}{f'\left( x \right) }, \quad (\forall f'\left( x_0 \right) \neq  0).
\end{equation}

\begin{example}
    \begin{equation}
        \left( \arcsin x \right)' =\frac{1}{\sin' \arcsin x } = \frac{1}{\cos\left( \arcsin x \right) } = \frac{1}{\sqrt{1-x^{2}}}.
    \end{equation}
\end{example}

\begin{example}
    \begin{equation}
      \left( \arccos x \right) ' = \frac{1}{\cos' \arccos x} = \frac{1}{- \sin \arccos x } = \frac{1}{- \sqrt{1-x^{2}}}
    \end{equation}
\end{example}

\begin{example}
    \begin{equation}
      \left( \arctan x \right) ' = \frac{1}{\tan ' \arctan x} = \cos^{2} \arctan x = \frac{1}{1 + x^{2}}
    \end{equation}
\end{example}

\subsection{复合函数的高阶导}
\begin{equation}
  \begin{gathered}
    h^{(1)} = g'\left( f\left( x \right)  \right) f'\left( x \right) \\
    h^{(2)} = g'' \left( f\left( x \right)  \right) f'\left( x \right) + g' \left( f\left( x \right)  \right) f''\left( x \right) \\
    h^{(3)} = g''' f' f' f' + g'' f'' f' + g'' f' f'' + g' f'''
  \end{gathered}
\end{equation}

\begin{definition}
    所谓$1, 2 , \ldots ,n$的一个分组方式$P = \{ A_1,A_2,\cdots,A_k \}$, 其中$A_1,A_2,\cdots,A_k$是$\{ 1,2,\cdots,n \}$的无交的非空子集, 且满足
    \begin{equation}
      A_1 \cup A_2 \cup  \cdots \cup A_k = \{ 1,2,\cdots,n \}
    \end{equation}.
\end{definition}
求导就是把求导算子分配到每一项的因式上.
\begin{theorem}
    \begin{equation}
      \left( g\circ f \right) ^{n} (x) = \sum_{\text{所有分组方案$P = \{A_1, \cdots, A_k\}$}}g^{(\text{组数})} f^{(|A_1|)} \cdots f^{(|A_k|)}
    \end{equation}
\end{theorem}

\begin{proof}
  采用归纳法, 几乎是显然的.
\end{proof}

\begin{example}
  $h\left( x \right) = \mathrm{e}^{\frac{\alpha}{2} x^{2}}$, 求$h^{(n)} (0)$.
  
  令
  \begin{equation}
    g(y) = \mathrm{e}^{y}, \quad f(x) = \frac{\alpha}{2} x^{2}, h\left( x \right) = \left( g\circ f \right) (x).
  \end{equation}

  经过计算可得
  \begin{equation}
    \left. \frac{\mathrm{d}^n}{\mathrm{d} x^n} \right|_{x = 0} \mathrm{e}^{ \frac{\alpha}{2} x^{2}} = \begin{cases} 
      0, & n\text{ 为奇数,} 
      \\ 
      \alpha ^{\frac{n}{2}} \cdot \left( n -1 \right) !!, & n \text{ 为偶数} 
    \end{cases}
  \end{equation}
\end{example}