% !TeX root = 高等微积分.tex

\paragraph{推论}
有限次四则运算和极限可交换.
\begin{equation}
    \lim_{k \to \infty}\sum_{i=1}^{n} x_{i,k} = \sum_{i=1}^{n} \lim_{k \to \infty} x_{i,k}
\end{equation}

\begin{equation}
    \lim_{k \to \infty}\left( \prod_{i=1}^{n} x_{i,k} \right) = \prod_{i[=1]}^{n} \left( \lim_{k \to \infty} x_{i,k} \right)  
\end{equation}

\begin{proof}
    只需$k-1$次使用前述定理.
\end{proof}
\paragraph{注意} 无限和/无限积与极限未必可交换.
\begin{equation}
    \lim_{k \to \infty}\sum_{i=1}^{\infty} x_{i,k} \neq \sum_{i=1}^{\infty} \left( \lim_{k \to \infty}x_{i,k} \right) 
\end{equation}
\begin{example}
   对于一个下表这样一个数列$x_{i,k}$,
   \begin{center}
    \begin{tabular}{c|c|c|c|c}
           &$k=1$&$k=2$  &$k=3$  &$\cdots$    \\ \hline
      $i=1$&$\frac{1}{1}$  &$\frac{1}{2}$  &$\frac{1}{3}$  &  \\ \hline
      $i=2$&$0$  &$\frac{1}{2}$  &$\frac{1}{3}$  &  \\ \hline
      $i=3$&$0$  &$0$  &$\frac{1}{3}$  &  \\ \hline
      $\vdots$
    \end{tabular}
  \end{center}
  纵向求和, 值是$1$, 但先取极限$k\to \infty$每一项都变为零, 再纵向求和, 值是$0$.
\end{example}

\textbf{类似地}, 有例子表明无限乘积与极限未必可交换.
\subsubsection{夹逼定理}
\begin{theorem}
    设$a_n\le b_n\le c_n \ (\forall n \ge N_0 )$, 如果
    \begin{equation}
        \lim_{n \to \infty}a_n = \lim_{n \to \infty} c_n = L
    \end{equation}
    则$\displaystyle \lim_{n \to \infty} b_n$存在且等于$L$.
\end{theorem}
\begin{proof}
  对于左右两边的数列极限,
  \begin{itemize}
    \item 由$\displaystyle \lim_{n \to \infty}a_n = L$定义可知, 
    \begin{equation}
      \exists N_1,\ \forall n\ge N_1, \text{有} \left| a_n - L \right| < \varepsilon
    \end{equation}
    从而
    \begin{equation}
      L -\varepsilon < a_n
    \end{equation}

    \item 由$\displaystyle \lim_{n \to \infty}c_n = L$定义可知, 
    \begin{equation}
      \exists N_2,\ \forall n\ge N_2, \text{有} \left| c_n - L \right| < \varepsilon
    \end{equation}
    从而
    \begin{equation}
      c_n < L+\varepsilon
    \end{equation}
  \end{itemize}
    
    结合起来, $\forall  n \ge \max \{ N_i \}$, 有 $L-\varepsilon < a_n \le b_n \le c_n < L+\varepsilon$.
\end{proof}

\begin{example}
    计算极限
    \begin{equation}
      \lim_{n \to \infty} \frac{a_k n^{k} + a_{k-1}n^{k-1} + \cdots + a_0}{n^{k}} = a_k
    \end{equation}
    因为
    \begin{align}
        \text{LHS} & = \lim_{n \to \infty}\left( a_k + \frac{a_{k-1}}{n} + \cdots + \frac{a_0}{n^{k}} \right) 
        \\
        & = \lim a_k + \lim \frac{a_{k-1}}{n} + \cdots 
        \\
        & = a_k + 0 + \cdots = a_k.
    \end{align}
\end{example}

\begin{example}
    \begin{align}
      & \lim_{n \to \infty} \frac{a_kn^{k}+ a_{k-1}n^{k-1} + \cdots + a_0}{b_l n^{l} + b_{l-1} n^{l-1} + \cdots +b_0}
      \\
      & = \lim_{n \to \infty}\left( \frac{a_kn^{k}+ \cdots +a_0}{n^{k}} \frac{n^{l}}{b_l n^{l}+ \cdots +b_0} n^{k-l} \right) 
      \\
      & = \begin{cases} 
        a_k \cdot \frac{1}{b_l}\cdot 0 = 0 , & k<l 
        \\ 
        a_k\cdot \frac{1}{b_l} \cdot 1 = 0, &  k=l
        \\
        \text{不存在(由引理)} , & k>l
      \end{cases}
    \end{align}
    \begin{lemma}
        设
        \begin{equation}
          \begin{aligned}
            & \lim_{n \to \infty}x_n = X \neq 0,
            \\
            & \lim_{n \to \infty} y_n = Y \neq 0,
            \\
            & \lim_{n \to \infty}Z_n \text{不存在},
          \end{aligned}
        \end{equation}
        则 
        \begin{equation}
          \lim_{n \to \infty} \left( x_n y_n z_n \right) \text{不存在}.
        \end{equation}
    \end{lemma}
    \begin{proof}
        反证法, 设$\displaystyle \lim_{n \to \infty}\left( x_n y_n z_n \right) = L$存在, 则
        \begin{align}
          \lim_{n \to \infty} z_n & = \lim_{n \to \infty} \left[ \left( x_n y_n z_n  \right) \frac{1}{x_n} \frac{1}{y_n} \right] 
          \\
          & = \lim_{n \to \infty} \left( x_n y_n z_n \right) \lim_{n \to \infty} \frac{1}{x_n} \lim_{n \to \infty} \frac{1}{y_n}
          \\
          & = L \cdot X \cdot Y.
        \end{align}
        与条件$\displaystyle \lim_{n \to \infty}z_n \text{不存在}$矛盾!
    \end{proof}
\end{example}

\begin{example}
  设$a_1,a_2,\cdots,a_k$是正数, 求
  \begin{equation}
    \lim_{n \to \infty} \left( a^n_1,a^n_2,\cdots,a^n_k \right)^{\frac{1}{n}}. 
  \end{equation}
  \textbf{解} 不妨设$a_1=\max \{ a_i \}$, 有
  \begin{equation}
    \left( a_1^{n} \right) ^{\frac{1}{n}} \le \left( a^n_1,a^n_2,\cdots,a^n_k \right) ^{\frac{1}{n}} \le \left( k a_1^{n} \right) ^{\frac{1}{n}}.
  \end{equation}
  注意到
  \begin{equation}
    \lim_{n \to \infty} \left( a_1^{n} \right) ^{\frac{1}{n}} = a_1, \ \lim_{n \to \infty}\left( k a_1^{n} \right) ^{\frac{1}{n}} = \lim_{n \to \infty} \sqrt[n]{k} a_1 = a_1.
  \end{equation}
  使用夹逼定理得到
  \begin{equation}
    \lim_{n \to \infty} \left( a^n_1,a^n_2,\cdots,a^n_k \right)^{\frac{1}{n}} = \max \{ a_i \}
  \end{equation}
\end{example}

\begin{example}
  进一步,
  \begin{align}
    & \lim_{n \to \infty} \left( a^{-n}_1,a^{-n}_2,\cdots,a^{-n}_k \right) ^{-\frac{1}{n}}
    \\
    = & \lim_{n \to \infty} \frac{1}{\left[ \left( \frac{1}{a_1} \right) ^{n} + \cdots + \left( \frac{1}{a_n} \right) ^{n} \right] ^{\frac{1}{n}}}
    \\
    = & \frac{1}{\max\left\{ \frac{1}{a_i} \right\}} = \frac{1}{1 / \min \{ a_i \}} 
    \\
    = & \min \{ a_i \}.
  \end{align}
\end{example}

\subsubsection[Stolz 定理]{计算极限的一个有用方法: Stolz theorem}
\begin{definition}
    称$\displaystyle \lim_{n \to \infty}x_n = +\infty$, 如果对$\forall k>0$,
    \begin{equation}
      \exists n\in \mathbb{Z}_{+} ,\ \forall n\ge N \text{有} x_n>k.
    \end{equation}
\end{definition}

\begin{theorem}[Stolz Theorem]
    设$\{ b_n \}$严格单调递增且无上界(或等价地说$\lim b_n = +\infty$).

    设$\displaystyle \lim_{n \to \infty} \frac{a_{n+1}-a_n}{b_{n+1}-b_n}=L$, 则
    \begin{equation}
      \lim_{n \to \infty} \frac{a_n}{b_n} = L.
    \end{equation}
\end{theorem}

\begin{proof}[证明 Stolz 定理]
    由$\displaystyle \lim_{n \to \infty} \frac{a_{n+1}-a_n}{b_{n+1}-b_n}=L$的定义可知, $\exists N \in \mathbb{Z}_{+},\ \forall n \ge N$有
    \begin{equation}
      L-\varepsilon< \frac{a_{i+1}-a_i}{b_{i+1}-b_i} < L+\varepsilon 
      \implies (L-\varepsilon)(b_{i+1}-b_{i}) < a_{i+1}-a_i < (L+\varepsilon)(b_{i+1}-b_i)
    \end{equation} 
    我们可以对上式对$i$从$N$到$n-1$求和, 得到
    \begin{gather}
      (L-\varepsilon)(b_{n}-b_{N}) < a_{n} - a_{N} < (L+\varepsilon)(b_{n}-b_{N})
      \\
      \mathop{\implies}\limits^{\text{除以}b_n}_{} 
      (L-\varepsilon)\left( 1- \frac{b_{N}}{b_n} \right) + \frac{a_{N}}{b_n} < \frac{a_n}{b_n} < (L+\varepsilon)\left( 1- \frac{b_{N}}{b_n} \right) + \frac{a_{N}}{b_n}.
    \end{gather}
    同时注意到
    \begin{equation}
      \lim_{n \to \infty} \frac{b_{N}}{b_n} = 0,\ 
      \lim_{n \to \infty} \frac{a_{N}}{b_n} = 0
    \end{equation}
    由于命题\ref{充分大指标的项保持极限不等式}``充分大指标的项保持极限不等式''
    , 可知$\exists N_0 \in \mathbb{Z}_{+}$, 使得$\forall n > N_0$都有
    \begin{equation}
      \left| \frac{a_n}{b_n}-L \right| < 2\varepsilon.
    \end{equation}
\end{proof}

\subsection[单调极限定理]{单调极限定理(Weierstrass定理)(Monotone Converge Theorem)}
\begin{theorem}[单调极限定理]
  有上界且递增的数列一定收敛; 有下界且递降的数列一定收敛.
\end{theorem}
\begin{proof}
  设$\{ x_i \}$递增且有上界, 考虑单步点集
  \begin{equation}
    X = \{ x_n | n\in \mathbb{Z}_{+} \},
  \end{equation}
  可知$X$非空且有上界, 由确界定理知, $\sup X$存在, 记为$L$.

  由$\sup X = L$的定义知,
  \begin{equation}
    \forall \varepsilon >0 ,\ L- \varepsilon \text{不是$X$上界},
  \end{equation}
  即$\exists  N\in \mathbb{Z}_{+}$, 使得$x_N > L-\varepsilon $, 从而对于
  $\forall n \ge  N$都有
  \begin{equation}
    L-\varepsilon < x_N \le  x_n \le  L,
  \end{equation}
  即
  \begin{equation}
    \left| x_n - L  \right| < \varepsilon.
  \end{equation}
  这表明$\displaystyle \lim_{n \to \infty}x_n = L$.
\end{proof}


\begin{theorem}[Euler]
    $\displaystyle \lim_{n \to \infty}\left( 1+\frac{1}{n} \right) ^{n}$存在(记为$\mathrm{e}^{}$).
\end{theorem}

\begin{proof}
    记$x_n = \left( 1+\frac{1}{n} \right) ^{n}$.
    \begin{itemize}
      \item $\{ x_n \}$有上界, 
      \begin{equation}
        \begin{aligned}
          x_n \le  & 1 + \frac{1}{1!} + \frac{1}{2!} + \cdots + \frac{1}{n!}
          \\
          < & 1 + \frac{1}{1} + \frac{1}{1\times 2} + \frac{1}{2\times 3} + \cdots + \frac{1}{\left( n-1 \right) \times n} <3
        \end{aligned}
      \end{equation}

      \item $\{ x_n \}$递增,
      \begin{equation}
        \sqrt[n+1]{x_n} = \sqrt[n+1]{\underbrace{\left( 1+\frac{1}{n} \right) \cdots \left( 1+\frac{1}{n} \right) }_{\text{$n$个}}\cdot 1} \le  \frac{\left( 1+\frac{1}{n} \right) + \cdots + \left( 1+\frac{1}{n} \right) +1}{n+1} = 1 + \frac{1}{n+1}.
      \end{equation}
      所以我们得到了
      \begin{equation}
        \left( 1+\frac{1}{n} \right) ^{n} \le \left( 1+\frac{1}{n+1} \right) ^{n+1}.
      \end{equation}
      由单调极限定理可知, 极限存在, 称为自然常熟$\mathrm{e}$.
    \end{itemize}
\end{proof}

