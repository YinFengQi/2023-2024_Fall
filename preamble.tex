\usepackage{graphicx}
\usepackage{geometry}
\usepackage{amsfonts,amssymb,amsthm,amsmath,physics,extarrows}
\usepackage{mathrsfs}
\usepackage{tikz}
\usepackage{hyperref}% 这样可以有超链接(可以点击翻页)
\usepackage{simpler-wick}% wick收缩
% \usepackage{arydshln}% 虚线

\usetikzlibrary{arrows.meta,decorations.markings,calc,bending}



\newtheorem{definition}{定义}[section]
\newtheorem{lemma}{引理}[section]
\newtheorem{theorem}{定理}[section]
\newtheorem{proposition}{命题}[section]
\newtheorem{example}{例}[section]

\newcommand{\bm}[1]{\boldsymbol{#1}}
%这行是用来支持latex-workshop预览的

\numberwithin{equation}{section}

% 来自Castel的图片插入代码
\usepackage{import}
\usepackage{pdfpages}
\usepackage{transparent}
\usepackage{xcolor}

\newcommand{\incfig}[2][1]{%
    \def\svgwidth{#1\columnwidth}
    \import{./figures/}{#2.pdf_tex}
}

% \pdfsuppresswarningpagegroup=1 这一行似乎只在pdftex里使用

