% !TeX root = 场论与凝聚态.tex

\section{Classical $U(1)$ Theory}

Assigning a $U(1)$ valued field on each vertex, and we can write the partition function as
\begin{equation}
  \mathcal{Z} = \left( \prod_{\text{vertex}} \int_{-\pi }^{\pi } \frac{\mathrm{d} \theta_{v}}{2\pi } \right) \prod_{\text{link}} W \left( \mathrm{e}^{\mathrm{i} \mathrm{d}  \theta_{l}} \right),
\end{equation}
where $W$ is a statistical weight periodic in $\theta$, and other symbolic conventions are used as in lattice exterior derivative.

An realization of this model is separated superconductors with Joseph coupling, as shown in Fig. \ref{fig:superconductor-of-joseph-interaction}.
\begin{figure}[ht]
    \centering
    \incfig{superconductor-of-joseph-interaction}
    \caption{Superconductor of Joseph Interaction}
    \label{fig:superconductor-of-joseph-interaction}
\end{figure}

This system acquires a $U(1)$ global symmetry\footnote{
  Also a $\mathbb{Z}_2$ symmetry, i.e., reflection for rotor or charge conjugation for superconductor.
}, invariant under transformation of $\mathrm{e}^{\mathrm{i}\alpha}$, in which $\alpha$ satisfies $\mathrm{d} \alpha _{l} = 0$. Note that $\alpha$ might be different in uncoupled areas.
In classical statistical mechanic this model is called \emph{linear sigma model}, and in condensed matter, called \emph{XY model}.

The $\theta$ is $2\pi $ periodic, thus its spectrum is a series of integer, inferring that the conjugate momentum of $\theta$ is integer-quantized --- the property of angular momentum.
A usual choice for $W(\theta)$ is 
\begin{equation}
  W = \exp \left( \frac{\cos \mathrm{d} \theta -1}{T} \right)
\end{equation}

\subsection{Low $T$ Expansion}

Under low temperature, we can expand the weight as (to the spirit of renormalization group)
\begin{equation}
  \exp \left( \frac{\cos  \mathrm{d} \theta_{l} - 1}{T} \right) \sim \exp \left( - \frac{\mathrm{d} \theta_{l}^{2}}{2T} \right) 
\end{equation}
which lost the $2\pi $ periodicity.

We can get the asymptotic behaviour of $\left< \theta_{v_0}\theta_{v_1} \right>$ correlation from the property of gapless system,
\begin{equation}
  \left< \theta_{v_0} \theta_{v_1} \right> \xlongrightarrow{r \to \infty }
  \begin{cases}
    r^{-\left( d-2 \right)}, & d>2, \\
    \frac{T}{2\pi } \ln r, & d=2.
  \end{cases}
\end{equation}

Now we consider how to relate this to the correlation of order parameter $\left< \mathrm{e}^{\mathrm{i}\theta_{v_0}} \mathrm{e}^{-\mathrm{i} \theta_{v_1}} \right>$. By Gaussian integral, we have
\begin{equation}
  \int \mathcal{D} [x]\, \mathrm{e}^{- \frac{1}{2} x A x} \mathrm{e}^{\mathrm{i} x} = \mathrm{e}^{- \frac{1}{2} A^{-1}}
\end{equation}
Thus we have
\begin{equation}
  \left< \mathrm{e}^{\mathrm{i}\left( \theta_{v_0} - \theta_{v_1} \right)} \right> = \mathrm{e}^{- \frac{1}{2} \left< \left( \theta_{v_0} - \theta_{v_1} \right)^{2} \right>} = \mathrm{e}^{-\left< \theta_{v_0}^{2} \right>}.
\end{equation}
Substituting the $\theta\theta$ correlation, we obtain
\begin{equation}
  \left< \mathrm{e}^{\mathrm{i}\theta_{v_0}} \mathrm{e}^{-\mathrm{i}\theta_{v_1}} \right> = 
  \begin{cases}
    \mathrm{e}^{-\left< \theta_{v_0}^{2} \right>}, & d>2,
    \\
    \propto r^{- \frac{T}{2\pi }}, & d=2.
  \end{cases}
\end{equation}
Directly we can identify there's no SSB in the quasi-long range $d=2$ case, and SSB appears in $d > 2$ case.
In conclusion, we have
\begin{itemize}
  \item continuous, compact global symmetry systems have SSB only in $d>=3$,
\end{itemize}

\section{High $T$ Expansion}

We make a Fourier transformation and obtain
\begin{equation}
  \exp \left( \frac{\cos \mathrm{d} \theta - 1}{T} \right) = \sum_{j \in \mathbb{Z}} \mathrm{e}^{-\frac{1}{T}} \cdot I_{j} \left( \frac{1}{T} \right)\mathrm{e}^{- \mathrm{i} j_{l} \left( \mathrm{d} \theta \right)_{l}},
\end{equation}
where $I_{j}$ stands for some Bessel function, whose relative magnitude relation takes the form of Fig. \ref{fig:plot-of-bessel-functions}.
\begin{figure}[ht]
    \centering
    \incfig{plot-of-bessel-functions}
    \caption{Plot of Bessel functions}
    \label{fig:plot-of-bessel-functions}
\end{figure}

We apply the Fourier transformation in the partition function, and we will also see that this step will turn the system into current representation later
\begin{equation}
  \mathcal{Z} = \left( \prod_{v}  \int _{-\pi }^{\pi } \frac{\mathrm{d} \theta_{l}}{2\pi } \right) \left( \prod_{l} \sum_{j \in \mathbb{Z}}  \right) \left( \mathrm{e}^{- 1 / T} I_{j}\left( \frac{1}{T} \right) \right) \left( \prod_{v} \mathrm{e}^{-\mathrm{i} \theta_{v} \left( \nabla \cdot j \right)_{v}} \right).
\end{equation}
Note that $\theta_{v}$ isn't involved in any differential operator doesn't have propagation mode, and hence only serve as a constrain multiplier. By integrating it out, the partition function gets a term of $\delta \left( \nabla  \cdot j \right)$ , implying that only current-conserved configs contribute.

Analogously, insertion of observables in the path integral works as creation and annihilation operators, i.e., correlators $\left< \mathrm{e}^{\mathrm{i}\theta_{v_0}} \mathrm{e}^{-\mathrm{i}\theta_{v_1}} \right>$ are composed by conserved current diagrams starting and ending at $v_0, v_1$. Now we skip a short discussion and put down the correlation function
\begin{equation}
  \left< \mathrm{e}^{\mathrm{i}\theta_{v_0}} \mathrm{e}^{-\mathrm{i}\theta_{v_1}} \right> \propto \left( I_1 \left( \frac{1}{T} \right) \right)^{r},
\end{equation}
which is short range as expected.
\begin{figure}[ht]
  \centering
  \incfig{xy-model-high-t-correlator}
  \caption{XY model high $T$ correlator}
  \label{fig:xy-model-high-t-correlator}
\end{figure}

\paragraph{A comment here} A quite abnormal phenomenon that this system under low $T$ expansion behaves like a free field theory, comes from the approach of our expansion. The term $\exp \left( \frac{\mathrm{d} \theta_{l} ^{2}}{2T} \right)$ have no response to $T$, as shown by simply rescaling $\theta_{v} \to \theta_{v} / \sqrt{T}$, meaning that, at any dimension, this model has no phase transition no matter how low the temperature is. Till now, readers might have realized that \emph{$2\pi $ periodicity cannot be ignored even at low $T$} and that it accounts for a large part of the physics of this system.

