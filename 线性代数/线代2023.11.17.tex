% !TeX root = 线性代数.tex

\subsubsection{奇异值分解的应用}
\begin{equation}
  A = U \Sigma V^{\mathrm{T}} = \sum_{i=1}^r \sigma_i u_i v_i^{\mathrm{T}}
\end{equation}
其中$U, V$为正交矩阵, $\Sigma$为对角矩阵, 对角线上的元素为奇异值$\sigma_i$.

\paragraph{矩阵近似}
我们定义矩阵的最大奇异值为其范数, $||A|| = \displaystyle \max_{x\neq  0} \frac{|Ax|}{|x|} = \sigma_{1}$ 在所有秩为$k$的矩阵当中, 有
\begin{equation}
  ||A-B|| \ge  ||A - A_k||, \quad A_k = \sum_{i=1}^k \sigma_i u_i v_i^{\mathrm{T}}.
\end{equation}
并且有$||A - A_k || = \sigma_{k+1}$.
\begin{proof}
    对于一个任意的秩为$k$的矩阵, 有分解
    \begin{equation}
      B = X Y^{\mathrm{T}}
    \end{equation}
    其中, $X$是一个$ m \times k$矩阵, $Y$是一个$n \times k$矩阵. 

    因为对于矩阵$A$, 存在一个可逆矩阵$P$, 使得$A = PR$, $R$为行约化阶梯形式, 因为$A$的秩为$k$, 所以$R$的后$m-k$行全为零, 所以有
    \begin{equation}
      R = \begin{bmatrix}
       R_1\\
       0\\
      \end{bmatrix}.
    \end{equation}
    把可逆矩阵$P$写为$P = \begin{bmatrix}
     P_1 & P_2\\
    \end{bmatrix}$, 这样有
    \begin{equation}
      A = PR = P_1 R_1.
    \end{equation}
    这样令$X = P_1$, $Y^{\mathrm{T}} = R_1$. 这样就得到了分解$A = XY^{\mathrm{T}}$.

    可以找到一个特殊向量,
    \begin{equation}
      w = \sum_{i=1}^{k+1} y_i v_i,
    \end{equation}
    使得$Y^{\mathrm{T}}w = 0$. (不妨令$y_1^{2} + y_2^{2}+ \cdots +y_{k+1}^{2} = 1$)这样就有
    \begin{equation}
        \begin{aligned}
            ||A-B||^{2} \ge |(A-B)w|^{2} & = |Aw|^{2} =w^{\mathrm{T}}A^{\mathrm{T}}Aw\\
            & = y_{1}^{2} \sigma_1^{2} + \cdots + y_{k+1}^{2} \sigma_{k+1}^{2} \\
            &\ge  \sigma_{k+1}^{2}
        \end{aligned}
    \end{equation}
\end{proof}

\paragraph{$QS$分解}
用奇异值分解可以得到一个$QS$分解
\begin{equation}
    \begin{aligned}
        A = U\Sigma V^{\mathrm{T}} & = U V^{\mathrm{T}}V \Sigma V^{\mathrm{T}} 
        \\
        & =\left( U V^{\mathrm{T}} \right) \left( V\Sigma V^{\mathrm{T}} \right) 
        \\
        & \equiv QS
    \end{aligned}
\end{equation}
其中$S$为一个对称矩阵, 且为一个半正定矩阵, $Q$为一个正交矩阵.

\begin{proof}[验证$S$为半正定矩阵]
    \begin{equation}
      x^{\mathrm{T}} S x = x^{\mathrm{T}} V \Sigma V^{\mathrm{T}} x = \left( V^{\mathrm{T}} x \right) ^{\mathrm{T}} \Sigma \left( V^{\mathrm{T}} x \right) \ge 0
    \end{equation}
\end{proof}

\paragraph{广义逆}
通过奇异值分解可以定义广义逆
\begin{equation}
  A^{+} = V \Sigma^{+} U^{\mathrm{T}},
\end{equation}
其中
\begin{equation}
  \Sigma^{+} = \begin{bmatrix}
   \sigma_1^{-1} &  &  &  & \\
    & \sigma_2^{-1} &  &  & \\
    &  & \ddots &  & \\
    &  &  & \sigma_r^{-1} & \\
    &  &  &  & 0\\
    \end{bmatrix}
\end{equation}
满足
\begin{equation}
    A A^{+} A = A, \quad A^{+} A A^{+} = A^{+}.
\end{equation}

对于一个不可逆的$A$, $Ax = b$有多个解, $x = A^{+}b$是最佳解, 且是满足投影方程$A^{\mathrm{T}}A x = A^{\mathrm{T}} b$的最短的向量.

\section{线性映射}
线性映射的性质
\begin{itemize}
    \item 把零向量映射到零向量
    \item 所有映射到零向量的在$x$中的向量称为核, 记为$\ker f$
    \item $f$在$Y$中的像称为值域, 记为$\Im f$
\end{itemize}