% !TeX root = 线性代数.tex

\section{特征值和特征向量}
考虑下列线性方程组(特征方程),
\begin{equation}
  A\vec{x}=\lambda \vec{x}.
\end{equation}
$\vec{x}$是一个非零的向量, 称为$A$的特征向量, 常数$\lambda$称为$A$的特征值.

\begin{example}
    求矩阵的特征向量和特征值
    $A = \begin{bmatrix}
     1 & 1\\
     0 & 1\\
    \end{bmatrix}$.
    \\
    \textbf{解}
    \ 
    特征方程可以写为$\left( A - \lambda I \right) \vec{x} = 0$, 即
    \begin{equation}
      \begin{bmatrix}
       1-\lambda & 1\\
       0 & 1-\lambda\\
      \end{bmatrix} \begin{bmatrix}
       x_1\\
       x_2\\
      \end{bmatrix} = \begin{bmatrix}
       0\\
       0\\
      \end{bmatrix}.
    \end{equation}
    得到
    \begin{equation}
      \begin{cases}
        \left( 1 - \lambda \right) x_1 + x_2 = 0
        \\
        \left( 1-\lambda \right) x_2 = 0
      \end{cases}
    \end{equation}
    所以只有$\lambda = 1$时才有非零解.
    此时, $\vec{x} = \begin{bmatrix}
     1\\
     0\\
    \end{bmatrix}$.
\end{example}

\begin{example}
    求矩阵的特征向量和特征值$ A = \begin{bmatrix}
     1 & 3\\
     1 & 2\\
    \end{bmatrix}$
    \\
    \textbf{解}
    \
    特征方程为$\left( A - \lambda I \right) \vec{x} = 0$, 即
    \begin{equation}
      \begin{bmatrix}
       1-\lambda & 3\\
       1 & 2-\lambda\\
      \end{bmatrix} \begin{bmatrix}
       x_1\\
       x_2\\
      \end{bmatrix} = \begin{bmatrix}
       0\\
       0\\
      \end{bmatrix}.
    \end{equation}
    得到
    \begin{equation}
          \begin{cases}
        \left( 1 - \lambda \right) x_1 + 3x_2 = 0
        \\
        x_1 + \left( 2-\lambda \right) x_2 = 0
      \end{cases}
    \end{equation}
    把第二个方程带入第一个方程, 有
    \begin{equation}
        \left[ \left( 1 - \lambda \right) \left( 2 - \lambda \right) -3 \right] x_2 = 0.
    \end{equation}

    所以$\lambda$有两个解.
\end{example}

考虑一般的特征方程, 
\begin{equation}
  \left( A - \lambda I \right) \vec{x} = 0.
\end{equation}
所以$A - \lambda I$的零空间维数大于等于一. 那么特征方程有非零解的充要条件就是
\begin{equation}
  \det \left( A - \lambda I \right) = 0.
\end{equation}

计算$\left| \lambda I - A \right| $,
\begin{equation}
    \left| \lambda I - A \right| = \begin{vmatrix}
     \lambda - a_{11} & -a_{12} & \cdots & -a_{1n}\\
     -a_{21} & \lambda - a_{22} & \cdots & -a_{2n}\\
     \vdots & \vdots & \ddots & \vdots\\
     -a_{n1} & -a_{n2} & \cdots & \lambda - a_{nn}\\
    \end{vmatrix}.
\end{equation}
这个行列式的值是一个关于$\lambda$的多项式, 称为特征多项式, 记为$p\left( \lambda \right)$.
我们可以发现, $\lambda$的最高次幂和次高次幂都来自于对角元的乘积.

我们把特征方程展开, 
\begin{equation}
  p\left( \lambda \right) = \lambda ^{n} + c_1 \lambda ^{n-1} + \cdots + c_n = 0.
\end{equation}
通过观察可以发现,
\begin{equation}
    c_1 = - \left( a_{11} + a_{22} + \cdots + a_{nn} \right) = - \tr A.
\end{equation}
\begin{equation}
    c_n = \left| -A \right| = \left( -1 \right) ^{n} \det A.
\end{equation}

特征多项式有唯一分解:
\begin{equation}
    p\left( \lambda \right) = \left( \lambda - \lambda_1 \right) \left( \lambda - \lambda_2 \right) \cdots \left( \lambda - \lambda_n \right).
\end{equation}
\begin{itemize}
    \item $\lambda_i$可能是复数.
    \item $\lambda_i$可能重合, 这一个特征值的代数重数为$\lambda_i$在上述分解中出现的次数.
    \item 特征值必定存在, 至少一个.
    \item 所有特征值的代数重数之和等于$n$.
\end{itemize}

\subsection{特征多项式的系数和特征值的关系}
\begin{equation}
    \begin{cases}
        c_1 = - \left( a_{11} + a_{22} + \cdots + a_{nn} \right) = - \tr A = - \left( \lambda_1 + \lambda_2 + \cdots + \lambda_n \right)
        \\
        c_n = \left| -A \right| = \left( -1 \right) ^{n} \det A = \left( -1 \right) ^{n} \lambda_1 \lambda_2 \cdots \lambda_n
    \end{cases}
\end{equation}

所以, 
\begin{equation}
  \sum_{i} \lambda_i = \tr A, \quad \prod_{i} \lambda_i = \det A.
\end{equation}

\subsection{特征值的一些简单性质}
\begin{itemize}
    \item 如果$\lambda$为$A$的特征值, 那么$\lambda^k$也是$A^k$的特征值, 因为
    \begin{equation}
        A^k \vec{x} = \lambda^k \vec{x}.
    \end{equation}

    \item 如果$A$可逆, 则$\lambda \neq 0$, 因为
    \begin{equation}
        A x = \lambda x \implies A^{-1} x = \frac{1}{\lambda} x.
    \end{equation}
\end{itemize}

\subsection{特征向量的一些简单性质}
\begin{itemize}
    \item 固定一个特征值, 所有对应的特征向量和零向量张成一个线性空间, 称为特征向量子空间, 记为$V\left( \lambda \right) $.
    \begin{enumerate}
        \item 加法下封闭: $\vec{x}_1, \vec{x}_2 \in V\left( \lambda \right) $
        \begin{equation}
          A \left( \vec{x}_1 + \vec{x}_2  \right) = \lambda \left( \vec{x}_1 + \vec{x}_2 \right) \implies \vec{x}_1 + \vec{x}_2 \in V\left( \lambda \right)
        \end{equation}
        \item 数乘下封闭: $\vec{x} \in V\left( \lambda \right) $
        \begin{equation}
          A \left( c \vec{x} \right) = c \lambda \vec{x} \implies c \vec{x} \in V\left( \lambda \right)
        \end{equation}
    \end{enumerate}

    \item 对于一个代数重数为$p$的特征值, 对应的特征向量子空间的维数称作几何重数, 满足 $\text{几何重数} \le \text{代数重数}$.
    
    \item 不同特征值对应的特征向量是线性无关的.
\end{itemize}

\begin{proof}
    只考虑两个特征值的情况, 设$\lambda_1, \lambda_2$为特征值, $x,y$为对应的特征向量, 我们需要证明方程$c_1 x + c_2 y = 0$只有零解.

    把$A$作用到这个方程, 得到
    \begin{equation}
        c_1 \lambda_1 x + c_2 \lambda_2 y = 0.
    \end{equation}
    这时候得到了两个方程, 消去$x$,
    \begin{equation}
      c_2 \left( \lambda_2 - \lambda_1 \right) y = 0.
    \end{equation}
    因为$\lambda_1 \neq \lambda_2$, 所以$c_2 = 0$, 同理$c_1 = 0$.
\end{proof}
