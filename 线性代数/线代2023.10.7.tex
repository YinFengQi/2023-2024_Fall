% !TeX root = 线性代数.tex

\subsection{用行约化阶梯形式求解线性方程组}
用行约化阶梯形式求解方程$Ax = b$, 方法如下

\begin{itemize}
    \item 考虑增广矩阵$[A|b]$, 做初等行变换, 把$A$变成行约化阶梯形式, 得到增广矩阵
    \begin{equation}
      [R|b']
    \end{equation}
    新的方程组$Rx = b'$的解空间和原来的方程一样.

    \item $Rx = b'$的解(如果存在)为
    \begin{equation}
      x=x_p+x_n
    \end{equation}
    其中$x_p$为$Rx = b'$的特解, $x_n$为对应的齐次线性方程组($b'=0$)的所有解.
    \begin{enumerate}
        \item $x_p$可以求解如下: 取自由变量为零, 主元变量任意, 可以得到一个解.
        \begin{example}
            \begin{equation}
              R = \begin{bmatrix}
               1 & 0 & 2 & 3\\
               0 & 1 & 0 & 1\\
               0 & 0 & 0 & 0\\
              \end{bmatrix}
              , \quad
              b' = \begin{bmatrix}
               2\\
               3\\
               0\\
              \end{bmatrix}
            \end{equation}
            特解为
            \begin{equation}
              x_p=\begin{bmatrix}
               2\\
               3\\
               0\\
               0\\
              \end{bmatrix}.
            \end{equation}
        \end{example}

        \item 齐次线性方程的解可以这样求:
        取某一个自由变量为$1$, 其他自由变量为$0$, 主元变量任意. 这样可以一共得到$n-r$个解, 记为$s_i$, $n$是变量数目, $r$是主元数目.
        则, 
        \begin{equation}
          x_n = a_1 s_1 +a_2s_2+ \cdots +a_n s_n.
        \end{equation}
        \begin{example}
            继续求解上例中的线性方程.
        
            第一个通解:
            \begin{equation}
              \begin{bmatrix}
               1 & 0 & 2 & 3\\
               0 & 1 & 0 & 1\\
               0 & 0 & 0 & 0\\
              \end{bmatrix}
              \begin{bmatrix}
               x_1\\
               x_2\\
               1\\
               0\\
              \end{bmatrix} = 0
            \end{equation}
            得到$x_1=-2, x_2=0$, 特解向量为$s = \begin{bmatrix}
             -2\\
             0\\
             1\\
             0\\
            \end{bmatrix}$.
            另一个通解:
            \begin{equation}
              \begin{bmatrix}
               1 & 0 & 2 & 3\\
               0 & 1 & 0 & 1\\
               0 & 0 & 0 & 0\\
              \end{bmatrix}
              \begin{bmatrix}
               x_1\\
               x_2\\
               0\\
               1\\
              \end{bmatrix}=0
            \end{equation}
            解向量为$s_2 = \begin{bmatrix}
             -3\\
             -1\\
             0\\
             1\\
            \end{bmatrix}$.
        
            于是原方程的所有解为
            \begin{equation}
              x = \begin{bmatrix}
               2\\
               3\\
               0\\
               0\\
              \end{bmatrix}
              + a_1 \begin{bmatrix}
                -2\\
                0\\
                1\\
                0\\
            \end{bmatrix}+ a_2 \begin{bmatrix}
                -3\\
                -1\\
                0\\
                1\\
            \end{bmatrix}.
        \end{equation}
    \end{example}
\end{enumerate}
\end{itemize}

\section{线性方程组的解}
\subsection{齐次线性方程解空间的性质}
\paragraph{基本性质}
\begin{itemize}
    \item 证明: 任何一个解都可以做上面的分解.

    $x'$为一个解, $x_p$为另一个解, 那么
    \begin{equation}
      \begin{cases} 
        Ax' = b
        \\ 
        A x_p = b 
      \end{cases}
    \end{equation}
    两式相减得到
    \begin{equation}
      A(x' -x_p)=0
    \end{equation}
    即, $x' = x_p +x_n$ 中的$x_n$是齐次线性方程的解.

    \item 反之, 对于任意的齐次线性方程的解$x_n$, $x_p + x_n$都是方程$Ax=b$的解.

    证明: 
    因为$\begin{cases} 
      A x_p=b 
      \\ 
      A x_n=0 
    \end{cases}
    \implies A(x_p+ x_n) = b$.
    
\end{itemize}

\paragraph{为什么$Ax=b$的解可以写成这种形式}

\begin{itemize}
    \item 如果 $v_1$ 为$Ax=0$的解, $v_2$也为解, 那么$v_1+v_2$也是方程的解.
    
    \item 如果$v$是一个解, 那么乘上一个系数$c$, $cv$也是方程的解.

    因为$Av=0$, 那么$A (cv) = c (Av) = 0$.
\end{itemize}

这证明了$Ax=0$的解空间$N(A)$在向量加法及数乘下是封闭的.

\subsection[一些概念]{线性子空间, 线性无关, 基, 维数}

\paragraph{线性子空间}
$\mathbb{R}^{m}$中的一个子空间$V$, 如果$V$在加法和数乘下面是封闭的, 那么这个子空间称为线性子空间.
\subparagraph{构造线性子空间的方法}
给定一组固定的向量$(v_1,v_2, \cdots ,v_n)$, 考虑所有的线性组合构成的空间
\begin{equation}
  V = \{ x_1 v_1 + x_2v_2 + \cdots + x_n v_n \}
\end{equation}
$V$是一个线性子空间, 称之为$(v_1, v_2, \cdots , v_n)$张成的线性子空间.

\paragraph{线性无关}
\begin{definition}
    一组向量$(v_1,v_2, \cdots, v_n)$称为线性无关的, 如果下列的方程
    \begin{equation}
      x_1v_1+x_2v_2+ \cdots  +x_n v_n = 0
    \end{equation}
    只有$0$解, 即对应的齐次线性方程$Ax=0$只有$0$解.
\end{definition}

\begin{example}
    $\mathbb{R}^{2}$中的$v_1 = \begin{bmatrix}
     1\\
     0\\
    \end{bmatrix}, v_2 = \begin{bmatrix}
     0\\
     1\\
    \end{bmatrix}$是线性无关的.
\end{example}

\begin{example}
    如果$0$向量在这组向量中, 那么这组向量是线性相关的.
\end{example}

\paragraph{线性空间的基}
\begin{definition}
    一组线性无关的向量$(e_1,e_2, \cdots , e_m)$称之为$V$的一组基, 如果$V$中任意一个向量都可以表示为这组向量的线性组合,
    \begin{equation}
      v = x_1e_1+x_2e_2+ \cdots  + x_n e_n.
    \end{equation}
    基中的向量个数称作维数.
\end{definition}

\begin{example}
    $\mathbb{R}^{2}$中的$e_1 = \begin{bmatrix}
     1\\
     0\\
    \end{bmatrix}, \quad e_2 = \begin{bmatrix}
     0\\
     1\\
    \end{bmatrix}$为一组基. 所以$\mathbb{R}^{2}$的维数为$2$.
\end{example}

\paragraph{基的几个重要性质}
\begin{itemize}
  \item 坐标唯一性: 给定一组基$(e_1,e_2,\cdots , e_m)$, 根据基的定义, 任意的向量都可以写成$$(e_1,e_2,\cdots , e_m)$$的线性组合, 即
  \begin{equation}
    v = a_1e_1+a_2e_2+ \cdots +a_m e_m.
  \end{equation}
  其中$(a_1,a_2,\cdots ,a_m)$称为$v$在基$(e_1,e_2,\cdots, e_m)$下的坐标.
  
  坐标是\textbf{唯一}的.
  \begin{proof}
      假设坐标不唯一, $v$可以有两种展开方式:
      \begin{equation}
        \begin{aligned}
          v & = a_1 e_1+a_2e_2+ \cdots +a_m e_m 
          \\
          v & = b_1 e_1+b_2e_2+ \cdots +b_m e_m 
        \end{aligned}
      \end{equation}
      两式相减得到
      \begin{equation}
        0 = (a_1-b_1)e_1+(a_2-b_2)e_2+ \cdots (a_m - b_m)e_m
      \end{equation}
      这与基的线性无关矛盾.
  \end{proof}

  \item 基不唯一, 但维数定义的维数一样.
  \begin{proof}
      反证法. 假设有两组基$(e_1,e_2,\cdots,e_m)$,$(f_1,f_2,\cdots,f_n)$, $n>m$.
       
      根据基的定义, $(f_1,f_2,\cdots,f_n)$可以写成$e_1,e_2,\cdots,e_m$的线性组合.
      \begin{equation}
        \begin{matrix}
          f_1 = a_{11}e_1
          + \cdots a_{m1} e_m
          \\
          \vdots
          \\
          f_n = a_{1n}e_1
          + \cdots a_{mn} e_m
        \end{matrix}
      \end{equation}
      把上述过程写成矩阵乘法的形式
      \begin{equation}
        F = \begin{bmatrix} f_1, f_2, \cdots, f_n \end{bmatrix},\quad E = \begin{bmatrix} e_1, e_2, \cdots, e_m \end{bmatrix}
      \end{equation}
      并且
      \begin{equation}
        F = EA,
      \end{equation}
      其中$A = (a_{ij})$, $A$为一个$m \times n$的矩阵.
  
      考虑$Ax = 0$的解, 利用之前齐次线性方程组的解的性质, 参数个数为$(n-r)$, $r$为主元数目, 且$r \le m$.
      所以$Ax=0$一定有非$0$的解($m\neq n$).
  
      利用方程$F=EA$, 如果$Ax=0$有非零解, 那么
      \begin{equation}
        Fx=EAx = 0
      \end{equation}
      也有非零解, 和假设矛盾.
  \end{proof}

  \item 基的变换矩阵$A$为可逆的.
  \begin{proof}
      有两组基$(f_1,f_2,\cdots,f_m), \ (e_1,e_2,\cdots,e_m)$, 
      \begin{equation}
        F = [f_1,f_2,\cdots,f_m], \quad E = [e_1,e_2,\cdots,e_m], \quad F=EA.
      \end{equation}
      $A$为$m \times m$矩阵. 如果$A$是不可逆的, 则$Ax=0$有非零解即
      \begin{equation}
        \begin{bmatrix}
          f_1, f_2, \cdots, f_m
        \end{bmatrix}
        \begin{bmatrix}
          x_1\\
          x_2\\
          \vdots\\
          x_m
        \end{bmatrix}
        = 0
      \end{equation}
      这与 $f_i$ 这组基的定义矛盾.
  \end{proof}
\end{itemize}

