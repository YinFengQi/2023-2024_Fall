% !TeX root = 费曼(3).tex

\chapter[晶格中的传播]{Propagation in Crystal}
\section[理想晶体]{Ideal Crystal}
After investigating double-state systems, we are able to study a system with discrete infinite energy levels.
Electrons on a $1$d lattice can be described by $\ket{n}$, under the approximation that each electron on vertices stays on its ground state, eliminating vagueness of this representation. Ignoring all tunneling between non-vicinity vertices, we have the Hamiltonian
\begin{equation}
  H \ket{n} = E_0 \ket{n} - A \ket{n-1} - A \ket{n+1}
\end{equation}
under the basis, the representation of $H$ is
\begin{equation}
  H = \begin{pmatrix}
   \ddots &  &  &  & \\
    & E_0 & -A &  & \\
    & -A & E_0 &-A  & \\
    &  & -A & E_0 & \\
    &  &  &  & &\ddots \\
  \end{pmatrix}
\end{equation}
Sch\"ordinger equation gives
\begin{equation}
  \mathrm{i}  \hbar \frac{\mathrm{d}C_n}{\mathrm{d} t} = E_0 C_n - A C_{n-1} - A C_{n+1}, \quad n\in \mathbb{Z}
\end{equation}
solution should take the form of $\mathrm{e}^{i\cdots}$. Denoting the position of the $n$th atom as $x_n = n b$, an ansats would be $C_n = \mathrm{e}^{- \mathrm{i} E t / \hbar }\mathrm{e}^{+ \mathrm{i} k x_n}$,
trivially, 
\begin{equation}
  \color{red}\boxed{\color{black}E = E_0 - 2A \cos kb}.
\end{equation}
the equation above is a dispersion relation. As long as our resolution of energy is greatly smaller than number of vertices, we can regard $k$ as a continuous variable on $\mathbb{R}$, but with periodicity. The first Brillouin area is $k \in \left(- \frac{\pi}{b}, \frac{\pi}{b}\right]$. Let us check the behavior under long wavelength approximation, namely $k$ is very small. 
\begin{equation}
  E \simeq E_0 - A b^{2} k^{2},
\end{equation}
which looks like the dispersion relation of non-relativistic free particle, whose \emph{effective mass} is $m_{\text{eff}} = \frac{\hbar^{2}}{2 A b^{2}}$. This degree of freedom is a quasi-particle.

For a $3$d orthogonal crystal, $E = E_0 - 2 \sum_{i=1}^{3} A_i \cos k_i b_i$, under long wave limit,
\begin{equation}
  E = E_0 + \sum_{i=1}^{3} \left( -2 A_i + A_i b_i^{2} k_i^{2} \right) 
\end{equation}
the effective mass might not be isotropic. But, in case $A_i b_i^{2}$ are all equal, an \emph{emergent symmetry}\footnote{It is related with renormalization group.} will appear.

\section[晶格缺陷]{Crystal Defect}
Let $H = H + F \ket{0} \bra{0}$, thus
\begin{equation}
    H = \begin{pmatrix}
        \ddots &  &  &  & \\
         & E_0 & -A &  & \\
         & -A & E_0 + F &-A  & \\
         &  & -A & E_0 & \\
         &  &  &  & &\ddots \\
       \end{pmatrix}
\end{equation}
suppose another ansats, split into a left incident wave and else ones
\begin{equation}
  C_n = \begin{cases} 
    \mathrm{e}^{\mathrm{i} k x_n} + \beta \mathrm{e}^{- \mathrm{i} k x_n}, & (n<0) 
    \\ 
    \gamma \mathrm{e}^{\mathrm{i} k x_n}, & (n>0) 
  \end{cases}
\end{equation}
