% !TeX root = 费曼(3).tex

\setcounter{chapter}{10}
\chapter[更多的双态系统]{More Double-State System}

Pauli matrices, 
\begin{equation}
  \sigma_{x} = \begin{bmatrix}
   0 & 1\\
   1 & 0\\
  \end{bmatrix}
  , \ 
  \sigma_{y} = \begin{bmatrix}
   0 & -\mathrm{i} \\
   \mathrm{i}  & 0\\
  \end{bmatrix}
  , \
  \sigma_{z} = \begin{bmatrix}
   1 & 0\\
   0 & -1\\
  \end{bmatrix}
\end{equation}
\begin{equation}
  \tr \sigma_i = 0,\ \det \sigma_i = -1,\ \sigma_i ^{\dagger} = \sigma_i.
\end{equation}
anti-commutator
\begin{equation}
    \{ \sigma_i, \sigma_j \} = 2 \delta_{ij}.
\end{equation}
commutator
\begin{equation}
  [\sigma_{i}, \sigma_j] = 2 i \epsilon_{ijk} \sigma_k.
\end{equation}
and we have
\begin{equation}
  \left( \vec{A} \cdot \vec{\sigma} \right) \left( \vec{B}\cdot \vec{\sigma} \right) = \vec{A}\cdot \vec{B} + \mathrm{i} \vec{\sigma} \cdot \left( \vec{A} \times \vec{B} \right).
\end{equation}

Spacial rotation
\begin{equation}
  \begin{gathered}
        D_1^{(1 / 2)} (\phi) = \begin{bmatrix}
            \cos \frac{\phi}{2} & \mathrm{i} \sin \frac{\phi}{2}\\
            \mathrm{i} \sin \frac{\phi}{2} & \cos \frac{\phi}{2}\\
        \end{bmatrix},\\
        D_2^{(1 / 2)} (\phi) = \begin{bmatrix}
            \cos \frac{\phi}{2} & \sin \frac{\phi}{2}\\
            -\sin \frac{\phi}{2} & \cos \frac{\phi}{2}\\
        \end{bmatrix},\\
        D_3^{(1 / 2)} (\phi) = \begin{bmatrix}
            \mathrm{e}^{-\mathrm{i} \phi / 2} & 0\\
            0 & \mathrm{e}^{\mathrm{i} \phi / 2}\\
        \end{bmatrix}.
  \end{gathered}
\end{equation}

Pauli matrices behave as components of a three-vector. The rotation matrix around x axis is 
\begin{equation}
  R_{x} \left( \phi \right) = \begin{bmatrix}
   1 & 0 & 0\\
   0 & \cos \phi & \sin \phi\\
   0 & -\sin \phi & \cos \phi\\
  \end{bmatrix}
\end{equation}
rotation of Pauli matrices is
\begin{equation}
  [\sigma_{i}]_{ab} = D \sigma_i D^{\dagger}.
\end{equation}
It is an invariant under three rotation.
