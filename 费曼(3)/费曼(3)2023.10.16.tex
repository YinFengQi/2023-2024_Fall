% !TeX root = 费曼(3).tex

\chapter[自旋$\frac{1}{2}$粒子]{Spin-$\frac{1}{2}$}

A system with spin-$\frac{1}{2}$ is a two-state system. And we may call it a qubit. The two states are usually denoted as $\ket{\uparrow}$ and $\ket{\downarrow}$. It is a building block of any spin system. 

\section[旋转的复合]{Combination of rotation}
Rotation is a form of transformation, under which the distance of 3D space is invariant. We only consider the rotations that can go to the identity transformation continuously(parity transformation is excluded). 

We use $R$ to denote a rotation operation. The combination is $R_1 R_2$, $R_2$ first and $R_1$ latter. The composite of two rotations is not commutable. We now need to find a map from the rotation to a matrix, such that the composite of two rotations is the product of the corresponding matrices. This is called a representation of the rotation.
\begin{equation}
  D_{ij}\left( R \right)  = D_{ij}\left( R_2 R_1 \right) = \sum_{k=1}^{3} D_{ik}\left( R_2 \right) D_{kj}\left( R_1 \right).
\end{equation}
where $D_{ij}$ is the matrix element of the representation. We have implicitly assumed that the representation is in 3-dimension.

The $D$ matrix is orthogonal, i.e. $D^T D = I$ and the determinant of $D$ is $1$.All $D$ matrix forms a group called $SO\left( 3 \right) $.


\section[投影表示]{Projection Representation}
A physical state is determined up to a uncertain phase, so the representation should satisfy
\begin{equation}
  D_{ij}\left( R_2R_1 \right) = \mathrm{e}^{\mathrm{i} \phi\left( R_2,R_1 \right) } \sum_{k} D_{ik}\left( R_2 \right) D_{kj} \left( R_1 \right) .
\end{equation}

We will use the convention of Euler angle. The rotation is a combination of three rotations,
\begin{equation}
    R\left( \gamma,\beta,\alpha \right) = R_z\left( \gamma \right) R_x\left( \beta \right) R_z\left( \alpha \right) .
\end{equation}
Now, in order to find the representation matrix, we will choose a basis for convenient. The basis is chosen as the eigenstates of $S_z$, i.e. $\ket{\uparrow}$ and $\ket{\downarrow}$. We know that a rotation along $z$-axis shouldn't change the state, thus it can only bring a phase factor.
Denote $C'_{j} = \bra{j T} \ket{\phi},\ C_{j} = \bra{j S}\ket{\phi}$, then we have
\begin{equation}
  \begin{cases} 
    \left| C'_{+} \right| = \left| C_{+} \right|
    \\ 
    \left| C'_{-} \right| = \left| C_{-} \right| .
  \end{cases}
\end{equation}
\begin{equation}
  \implies C'_{+} = \mathrm{e}^{\mathrm{i} \lambda} C_{+},\ C'_{-} = \mathrm{e}^{\mathrm{i} \mu} C_{-}.
\end{equation}
Eliminating a phase factor of the whole system, we have
\begin{equation}
  C'_{\pm } = \mathrm{e}^{\pm \mathrm{i} \frac{\lambda - \mu}{2}} C_{\pm }.
\end{equation}
Now we redefine $\frac{\lambda - \mu}{2} \rightarrow\lambda$.
According to the product rule of the representation matrix, we have
\begin{equation}
  \mathrm{e}^{\mathrm{i} \lambda \left( \phi_1 \right) } \mathrm{e}^{\mathrm{i} \lambda \left( \phi_2 \right) } = \mathrm{e}^{\mathrm{i} \lambda \left( \phi_1 + \phi_2 \right) } .
\end{equation}
thus, $\lambda$ depend on $\phi$ linearly. And $\lambda \left( \phi = 0 \right) $ should be zero.
\begin{equation}
  \implies \lambda = m \phi,\ m \in \mathbb{R}.
\end{equation}
Notice that the state should come back to itself after a $2\pi$ rotation, and shouldn't go back at $\pi$, thus
\begin{equation}
  m = \frac{1}{2}.
\end{equation}

Notice the relation between $z$-rotation and $x$-rotation,
\begin{equation}
  R_x \left( \beta \right) = R_y \left( -\frac{\pi}{2} \right) R_z\left( \alpha \right) R_y\left( \frac{\pi}{2} \right) .
\end{equation}

We now have to figure out the $\frac{\pi}{2}$ rotation along $y$-axis. We know that 
\begin{equation}
  D \left[ R_z\left( \phi \right)  \right] = \begin{pmatrix}
   \mathrm{e}^{\mathrm{i} \frac{\phi}{2}} & \\
    & \mathrm{e}^{-\mathrm{i} \frac{\phi}{2}}\\
  \end{pmatrix}
\end{equation}

Consider a rotation of $\pi$ along $y$-axis, we have
\begin{equation}
  C'_{+} = \mathrm{e}^{\mathrm{i} \beta} C_{-} ,\ C'_{-} = \mathrm{e}^{\mathrm{i} \gamma} C_{+}.
\end{equation}
meanwhile, for rotation of $2\pi$ along $y$-axis, which should equivalent to identity transformation, we have
\begin{equation}
    \beta + \gamma = 2\pi n,\ n \in \mathbb{Z}. 
\end{equation}
we define that $\gamma \equiv 0$, then
\begin{equation}
  D \left[ R_y\left( \pi \right)  \right] = \begin{pmatrix}
   0 & 1\\
   -1 & 0\\
  \end{pmatrix}.
\end{equation}